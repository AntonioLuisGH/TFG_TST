\chapter*{\notationname}
\pagestyle{especial}
\chaptermark{\notationname}
\phantomsection
\addcontentsline{toc}{listasf}{\notationname}
%\section*{Notación}
%\begin{table}[htbp]
\begin{longtable}{p{3cm}p{8.5cm}}

%$\displaystyle D$ & Tasa de símbolos  (sim/s) \\
%$\displaystyle R_b$ & Tasa binaria (bit/s) \\
%$\displaystyle T$ & Tiempo de símbolo (s) \\
%$\displaystyle T_{b}$ & Tiempo de bit (s) \\
%$W\left( {t} \right)$ & Ruido blanco\\
%$w\left( {t} \right)$ & Función muestra de un ruido blanco\\
%$\displaystyle h_{c}\left( {t} \right)$ & Respuesta impulsiva de un canal LTI continuo en el tiempo\\
%$\displaystyle H_{c}\left( {\omega} \right)$ & Respuesta en frecuencia de un canal LTI continuo en el tiempo\\
%$\displaystyle h_{c}\left( {\tau;t} \right)$ & Respuesta impulsiva de un canal LTV continuo en el tiempo\\
%$\displaystyle H_{c}\left( {\omega;t} \right)$ & Respuesta en frecuencia de un canal LTV continuo en el tiempo\\
%$\displaystyle h_{c}\left( {n} \right)$ & Respuesta impulsiva de un canal LTI discreto en el tiempo\\
%$\displaystyle H_{c}\left( {\Omega} \right)$ & Respuesta en frecuencia de un canal LTI discreto en el tiempo\\
$\RR$ & Cuerpo de los números reales \\
$\CC$ & Cuerpo de los números complejos\\
$\left\| \vc{v} \right\|$ & Norma del vector $\vc{v}$ \\
$\left\langle {\vc{v}, \vc{w}} \right\rangle$ & Producto escalar de los vectores $\vc{v}$ y $\vc{w}$\\
$\left| {\vc{A}} \right|$ &Determinante de la matriz cuadrada $\vc{A}$\\
$\textrm{det}\left( {\vc{A}} \right)$ &Determinante de la matriz (cuadrada) $\vc{A}$\\
$\vc{A}\trs$ & Transpuesto de $\vc{A}$\\
$\vc{A}\inv$ & Inversa de la matriz $\vc{A}$\\
$\vc{A}{\psd}$ & Matriz pseudoinversa de la matriz $\vc{A}$\\
$\vc{A}\her$ & Transpuesto  y conjugado de $\vc{A}$\\
$\vc{A}\cnj$ & Conjugado\\
c.t.p. & En casi todos los puntos\\
c.q.d. & Como queríamos demostrar\\
\ensuremath{\blacksquare}& Como queríamos demostrar\\
\ensuremath{\square}& Fin de la solución\\
e.o.c. & En cualquier otro caso\\
$\e$ & número e\\
$\xp{x}$ & Exponencial compleja\\
$\xppi{x}$ & Exponencial compleja con $2\pi$\\
$\nxp{x}$ & Exponencial compleja negativa\\
$\nxppi{x}$ & Exponencial compleja negativa con $2\pi$\\
$\re$ & Parte real\\
$\im$ & Parte imaginaria\\
$\sen$ & Función seno \\
$\tg$ & Función tangente \\
$\arctg$ & Función arco tangente \\
$\sento{y}{x}$ & Función seno de $x$  elevado a $y$\\
$\costo{y}{x}$ & Función coseno de $x$  elevado a $y$\\
$\sa$ & Función sampling \\
$\sgn$ & Función signo \\
$\rect$ & Función rectángulo \\
$\sinc$ & Función sinc\\
$\pder{y}{x} $ & Derivada parcial de $y$ respecto a $x$\\
$x\gra$ & Notación de grado, $x$ grados.\\
%
%$C_{XY}$& covarianza  de dos variables aleatorias reales $X$ e $Y$\\
%$R_{XY}$& correlación  de dos variables aleatorias reales $X$ e $Y$\\
%$\rho_{XY}$ &Coeficiente de correlación de las variables aleatorias reales $X$  e $Y$\\
%$\vc{Z}$ & Vector aleatorio complejo\\
%$\displaystyle F_{X}\left( {\cdot} \right)$ & Función de distribución de la variable aleatoria $X$ \\
%$\displaystyle f_{X}\left( {\cdot} \right)$ & Función densidad de probabilidad de la variable aleatoria $X$ \\
%$p_{X}\left( {\cdot} \right)$ & Función masa de probabilidad de la variable aleatoria discreta $X$ \\
%
$\Pr\left( {A} \right)$ & Probabilidad del suceso $A$ \\
$\displaystyle E\left[ {X} \right]$ & Valor esperado de la variable aleatoria $X$ \\
$\si{X}$ & Varianza de la variable aleatoria $X$\\
$\sim f_{X}\left( {x} \right)$ & Distribuido siguiendo la función densidad de probabilidad $f_{X}\left( {x} \right)$\\
%
$\gauss{m_{X}}{\si{X}}$ &Distribución gaussiana para la variable aleatoria X, de media $m_{X}$ y varianza $\si{X}$ \\
$\id{n}$ & Matriz identidad de dimensión $n$\\
$\diag{\vc{x}}$ & Matriz diagonal a partir del vector $\vc{x}$\\
$\diag{\vc{A}}$ & Vector diagonal de la matriz $\vc{A}$\\
$\snr$& Signal-to-noise ratio \\
$\mse$ & Minimum square error\\
$\talq$ & Tal que \\
$\eqdef$ & Igual por definición \\
$\norm{\vc{x}}$ & Norma-2 del vector $\vc{x}$\\
$\card{\vc{{A}}}$ & Cardinal, número de elementos del conjunto $\vc{A}$\\
$\xyz{\vc{x}}{i}{n}$ & Elementos $i$, de 1 a $n$, del vector $\vc{x}$\\
%\newcommand{\xyz}[3]{\ensuremath{#1_{#2},#2=1,2,\ldots,#3}}
$\df{x}$& Diferencial de $x$\\
$\le$ & Menor o igual \\
$\ge$ & Mayor o igual \\
$\BL$ & Backslash \\
$\iff$ & Si y sólo si \\
$x=a+3\eqexpl{a=1} 4 $& Igual con explicación \\
$\tfrac{a}{b}$ & Fracción con estilo pequeño, $a/b$ \\
$\inc$ & Incremento \\
$b\ten{a}$ & Formato científico \\
$\tendsub{x}$ & Tiende, con x \\
$\ord$ & Orden\\
$\tm$ & Trade Mark\\
$\E[x]$ & Esperanza matemática de x\\
$\covm{\vc{x}}$ & Matriz de covarianza de $\vc{x}$\\
$\corrm{\vc{x}}$ & Matriz de correlación de $\vc{x}$\\
$\si{x}$ & Varianza de x \\


\end{longtable}
\newpage
%\end{table}
%


%\phantomsection
%\addcontentsline{toc}{listasf}{Acrónimos}
%\section*{Acrónimos}
%\begin{table}[htbp]
%\begin{tabular}{p{2cm}p{10cm}}
%Escuela Técnica Superior de In
%LTI & Lineal Invariante con el Tiempo \\
%LTV& Lineal Variable con el Tiempo\\
%AWGN& Ruido blanco gaussiano aditivo\\
%DMS& Fuente discreta sin memoria\\
%AEP& Propiedad de equipartición asintótica\\
%WLLN& Ley Débil de los Grandes Números\\
%DMC& Canal Discreto sin Memoria\\
%BSC& Canal Simétrico Binario\\
%BEC& Canal Binario con Borrado\\
%\end{tabular}
%\end{table}


%\nota{El libro de Lapidoth tiene una excelente recopilación.}