% !TEX root =../LibroTipoETSI.tex
\chapter*{Prefacio}
\pagestyle{especial}
\chaptermark{Prefacio}
\phantomsection
\addcontentsline{toc}{listasf}{Prefacio}

%\lettrine[lraise=0.7, lines=1, loversize=-0.25]{E}{n} %Para arriba
\lettrine[lraise=-0.1, lines=2, loversize=0.25]{E}{l} presente texto intenta explicar brevemente como utilizar la plantilla de latex para maquetar sus documentos. Encontrará un primer capítulo con aspectos generales sobre el uso de la hoja de estilos. El segundo capítulo es muy útil a la hora de recurrir a un ejemplo de uso de los elementos más utilizados. El tercer y último capítulo es de interés para aquellos autores que quieran redactar un libro de problemas resueltos, pudiendo intercalar secciones de teoría con secciones de problemas. También se incluyen ejemplos de Apéndices.

{\flushleft{\hfill \emph{Sevilla, 2012}}}%